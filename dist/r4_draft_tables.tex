
\begin{xltabular}{\linewidth}{X}
\caption{\label{tbl:physical}} \\
\toprule
診察項目 \\
\midrule
\endhead
頭部(顔貌、頭髪、頭皮、頭蓋)の診察 \\
眼(視野、瞳孔、対光反射、眼球運動・突出、結膜)の診察 \\
耳(耳介、聴力)の診察 \\
耳鏡を用いた外耳道、鼓膜の観察 \\
口唇、口腔、咽頭、扁桃の診察 \\
副鼻腔の診察 \\
鼻鏡を用いた前鼻腔の観察 \\
甲状腺、頸部血管、気管、唾液腺の診察 \\
頭頸部リンパ節の診察 \\
胸部の視診、触診、打診 \\
呼吸音と副雑音の聴診 \\
心音と心雑音の聴診 \\
腹部の視診、聴診(腸雑音、血管雑音)、打診、触診 \\
背部の叩打痛 \\
直腸(前立腺を含む)指診 \\
乳房の診察 \\
意識レベルの判定 \\
脳神経系の診察 \\
眼底検査 \\
腱反射の診察 \\
小脳機能・運動系の診察 \\
感覚系(痛覚、温度覚、触覚、深部感覚)の診察 \\
髄膜刺激所見 \\
四肢と脊柱(弯曲、疼痛)の診察 \\
関節(可動域、腫脹、疼痛、変形)の診察 \\
筋骨格系の診察(徒手筋力テスト) \\
\bottomrule
\end{xltabular}



\begin{xltabular}{\linewidth}{X}
\caption{\label{tbl:labo}} \\
\toprule
検査項目 \\
\midrule
\endhead
血算 \\
生化学検査 \\
凝固・線溶検査 \\
免疫血清学検査 \\
尿検査 \\
便検査 \\
 \\
血液型(ABO、RhD)検査、血液交差適合(クロスマッチ)試験、不規則抗体検査 \\
動脈血ガス分析 \\
妊娠反応検査 \\
細菌学検査(細菌の塗抹、培養、同定、薬剤感受性試験) \\
脳脊髄液 \\
胸水検査 \\
腹水検査 \\
病理組織検査や細胞診検査(術中迅速診断を含む) \\
 \\
染色体・遺伝子検査 \\
心電図 \\
呼吸機能検査 \\
内分泌・代謝機能検査 \\
脳波検査 \\
超音波検査 \\
X線撮影 \\
CT検査 \\
MRI検査 \\
核医学検査 \\
内視鏡検査 \\
\bottomrule
\end{xltabular}



\begin{xltabular}{\linewidth}{XXX}
\caption{\label{tbl:skills}} \\
\toprule
分類 & 基本的臨床手技 & 目標レベル \\
\midrule
\endhead
一般手技 & 体位交換、移送 & 実施できる \\
 & 気道内吸引 & 実施できる \\
 & 静脈採血 & 実演できる \\
 & 末梢静脈の血管確保 & 実演できる \\
 & 動脈血採血 & 実演できる \\
 & 腰椎穿刺 & 実演できる \\
 & 胃管の挿入と抜去 & 実演できる \\
 & 尿道カテーテルの挿入と抜去 & 実演できる \\
 & 皮内注射 & 実演できる \\
 & 皮下注射 & 実演できる \\
 & 筋肉注射 & 実演できる \\
 & 静脈内注射 & 実演できる \\
検査手技 & 微生物学検査(Gram 染色を含む) & 実施できる \\
 & 12 誘導心電図の記録 & 実施できる \\
 & 臨床判断のための簡易エコー(FAST含む) & 実演できる \\
 & 病原体抗原の迅速検査 & 実演できる \\
 & 簡易血糖測定 & 実演できる \\
外科手技 & 清潔操作 & 実演できる \\
 & 手術や手技のための手洗い & 実施できる \\
 & 手術室におけるガウンテクニック & 実施できる \\
 & 基本的な縫合と抜糸 & 実演できる \\
\bottomrule
\end{xltabular}



\begin{xltabular}{\linewidth}{X}
\caption{\label{tbl:departments}} \\
\toprule
診療科 \\
\midrule
\endhead
総合診療科 \\
救急科 \\
内科 \\
外科 \\
小児科 \\
産婦人科 \\
精神科 \\
皮膚科 \\
整形外科 \\
眼科 \\
耳鼻咽喉科 \\
泌尿器科 \\
脳神経外科 \\
放射線科 \\
麻酔科 \\
病理診断科 \\
臨床検査科 \\
形成外科 \\
リハビリテーション科 \\
歯科口腔外科 \\
\bottomrule
\end{xltabular}



\begin{xltabular}{\linewidth}{XX}
\caption{\label{tbl:symptoms}} \\
\toprule
主要症候 & 検討すべき鑑別疾患 \\
\midrule
\endhead
発熱 & 髄膜炎,上気道炎,扁桃炎,肺炎,結核,急性副鼻腔炎,尿路感染症,胆嚢炎,胆管炎,薬剤性,インフルエンザ,蜂巣炎,感染性心内膜炎 \\
全身倦怠感 & 結核,肝炎,心不全,うつ病,甲状腺機能低下症,鉄欠乏性貧血 \\
食思(欲)不振 & 消化性潰瘍,急性肝炎,うつ病,急性副腎不全 \\
体重減少 & 心不全,ネフローゼ症候群,甲状腺機能低下症 \\
体重増加 & 悪性腫瘍全般,糖尿病,甲状腺機能亢進症,うつ病,慢性閉塞性肺疾患<COPD>,神経性食思<欲>不振症<拒食症> \\
意識障害 & くも膜下出血,頭蓋内血腫,脳炎,脳出血,脳梗塞,髄膜炎,薬物中毒,アルコール性中毒,てんかん,心筋梗塞,急性大動脈解離,急性消化管出血,敗血症,低血糖,ショック,CO2ナルコーシス,ナトリウム代謝異常 \\
失神 & 不整脈,弁膜症(大動脈弁膜症),てんかん,肺塞栓症 \\
けいれん & 脳梗塞,脳出血,脳炎,脳症,熱性けいれん,てんかん \\
めまい & 良性発作性頭位めまい症,脳出血,脳梗塞,Meniere病,前庭神経炎 \\
浮腫 & 深部静脈血栓症,心不全,ネフローゼ症候群,慢性腎臓病,肝硬変,甲状腺機能低下症,薬剤性,リンパ浮腫,血管性浮腫 \\
発疹 & ウイルス性発疹症(麻疹),ウイルス性発疹症(風疹),ウイルス性発疹症(水痘),ウイルス性発疹症(ヘルペス),蕁麻疹,薬疹,皮膚炎(アトピー性皮膚炎),湿疹,結節性紅斑,伝染性紅斑,帯状疱疹 \\
咳・痰 & 上気道炎,副鼻腔炎,気管支炎,肺炎,肺結核,肺癌,間質性肺疾患,薬剤性,気管支喘息,アレルギー性鼻炎,胃食道逆流症<GERD>,感冒<かぜ症候群>,百日咳 \\
血痰・喀血 & 肺結核,肺癌,気管支拡張症 \\
呼吸困難 & 肺塞栓症,急性呼吸促(窮)迫症候群<ARDS>,気管支喘息,慢性閉塞性肺疾患<COPD>,肺炎,間質性肺疾患,肺結核,緊張性気胸,自然気胸,心不全,アナフィラキシー,急性喉頭蓋炎,窒息 \\
胸痛 & 肺塞栓症,気胸,急性冠症候群,急性大動脈解離,大動脈瘤破裂,パニック障害,帯状疱疹,胸膜炎,急性心膜炎 \\
動悸 & 不整脈,甲状腺機能亢進症,鉄欠乏性貧血,二次性貧血,パニック障害,不安障害 \\
嚥下困難 & 脳出血,脳梗塞,扁桃炎,食道癌 \\
腹痛 & 消化性潰瘍,機能性ディスペプシア<FD>,急性胃腸炎,急性虫垂炎,便秘症,汎発性腹膜炎,過敏性腸症候群,腸閉塞,腸重積症,鼠径ヘルニア,胆嚢炎,胆石症,急性膵炎,急性冠症候群,急性大動脈解離,流・早産,卵巣嚢腫(捻転),卵巣癌(捻転),子宮内膜症,尿路結石,憩室炎,虚血性大腸炎,腸間膜動脈塞栓症,異所性妊娠,糖尿病性ケトアシドーシス \\
悪心・嘔吐 & 急性胃腸炎,急性虫垂炎,腸閉塞,食中毒,脳出血,片頭痛,くも膜下出血,頭蓋内血腫,髄膜炎,急性心筋梗塞,妊娠,糖尿病性ケトアシドーシス,カルシウム代謝異常 \\
吐血 & 食道静脈瘤,胃,消化性潰瘍,胃癌,Mallory-Weiss症候群 \\
下血 & 消化性潰瘍,炎症性腸疾患,大腸癌,痔核,裂肛,虚血性大腸炎,憩室出血 \\
便秘 & 便秘症,過敏性腸症候群,甲状腺機能低下症,薬剤性,Parkinson病,腸閉塞,大腸癌 \\
下痢 & 急性胃腸炎,炎症性腸疾患,過敏性腸症候群,甲状腺機能亢進症,薬剤性 \\
黄疸 & 急性肝炎,慢性肝炎,肝硬変,肝癌,胆管炎,膵癌,薬剤性,溶血性貧血,薬剤性,胆管癌,生理的黄疸 \\
腹部膨隆・腫瘤 & 腸閉塞,鼠径ヘルニア,肝硬変,妊娠 \\
リンパ節腫脹 & 扁桃炎,ウイルス性発疹症(風疹),結核,悪性リンパ腫,その他の悪性腫瘍全般,伝染性単核{球}症 \\
尿量・排尿の異常 & 糖尿病,薬剤性,蓄尿障害,尿路感染症,前立腺肥大症,過活動膀胱,神経因性膀胱 \\
血尿 & 糸球体腎炎症候群,腎細胞癌,尿路結石,尿路感染症,膀胱癌 \\
月経異常 & 妊娠,薬剤性,月経困難症,子宮内膜症,子宮体癌,更年期障害 \\
不安・抑うつ & うつ病,双極性障害,不安障害,甲状腺機能亢進症,悪性腫瘍全般,認知症,Parkinson病,甲状腺機能低下症,悪性腫瘍全般,薬剤性,適応障害 \\
認知障害 & 脳梗塞,認知症,Parkinson 病,うつ病,甲状腺機能低下症,薬剤性,正常圧水頭症,慢性硬膜下血腫 \\
頭痛 & 緊張型頭痛,片頭痛,薬剤性,髄膜炎,脳出血,くも膜下出血,緑内障,急性副鼻腔炎,群発頭痛,巨細胞性動脈炎<側頭動脈炎> \\
運動麻痺・筋力低下 & 脳梗塞,一過性脳虚血発作,脳出血,頭蓋内血腫,てんかん,脊髄損傷,椎間板ヘルニア,多発性筋炎,皮膚筋炎,筋萎縮性側索硬化症,Guillain-Barre症候群,カリウム代謝異常 \\
歩行障害 & 脳出血,頭蓋内血腫,脳梗塞,Parkinson病,変形性脊椎症,,脊柱管狭窄症,椎間板ヘルニア,変形性関節症,骨折 \\
感覚障害 & 脊柱管狭窄症,椎間板ヘルニア,糖尿病,多発神経炎 \\
腰背部痛 & 急性大動脈解離,急性膵炎,膵癌,尿管結石,椎間板ヘルニア,変形性脊椎症,脊柱管狭窄症,脊椎圧迫骨折,急性腰痛症,化膿性脊椎炎 \\
関節痛・関節腫脹 & 痛風,外傷,関節リウマチ,全身性エリテマトーデス<SLE>,偽痛風,反応性関節炎,化膿性関節炎 \\
\bottomrule
\end{xltabular}



\begin{xltabular}{\linewidth}{XXX}
\caption{\label{tbl:知識}} \\
\toprule
臓器 & 分類 & 項目名 \\
\midrule
\endhead
血液・造血器・リンパ系 & 構造と機能 & 骨髄の構造 \\
 &  & 造血幹細胞から各血球への分化と成熟の過程 \\
 &  & 主な造血因子(エリスロポエチン、顆粒球コロニー刺激因子(G-CSF)、トロンボポエチン) \\
 &  & 脾臓、胸腺、リンパ節、扁桃とPeyer板の構造と機能 \\
 &  & 血漿タンパク質の種類と機能 \\
 &  & 赤血球とヘモグロビンの構造と機能 \\
 &  & 白血球の種類と機能 \\
 &  & 血小板の機能と止血や凝固・線溶の機序 \\
 & 症候 & 発熱 \\
 &  & 全身倦怠感 \\
 &  & 黄疸 \\
 &  & リンパ節腫脹 \\
 &  & 貧血 \\
 &  & 出血傾向 \\
 &  & 血栓傾向 \\
 & 検査方法 & 末梢血塗抹 \\
 &  & 凝固・線溶・血小板機能検査 \\
 &  & 骨髄検査(骨髄穿刺、骨髄生検) \\
 &  & 輸血関連検査 \\
 &  & タンパク分画、免疫電気泳動 \\
 &  & 遺伝子・染色体検査 \\
 & 特異的治療法 & 輸血 \\
 &  & 造血幹細胞移植 \\
神経系 & 構造と機能 & 中枢神経系と末梢神経系の構成 \\
 &  & 脳の血管支配と血液脳関門 \\
 &  & 脳のエネルギー代謝の特徴 \\
 &  & 主な脳内神経伝達物質(アセチルコリン・ドパミン・ノルアドレナリン)とその作用 \\
 &  & 髄膜・脳室系の構造と脳脊髄液の産生と循環 \\
 &  & 脊髄の構造、機能局在と伝導路 \\
 &  & 脊髄反射(伸張反射、屈筋反射)と筋の相反神経支配 \\
 &  & 脊髄神経と神経叢(頸・腕・腰仙骨)の構成および主な骨格筋支配と皮膚分布(デルマトーム) \\
 &  & 脳幹の構造と機能、および伝導路 \\
 &  & 脳神経の名称、核の局在、走行・分布と機能 \\
 &  & 大脳の構造と大脳皮質の機能局在(運動野・感覚野・言語野) \\
 &  & 辺縁系の構造と記憶・学習の機序との関連 \\
 &  & 錐体路を中心とした随意運動の発現機構 \\
 &  & 小脳の構造と機能 \\
 &  & 大脳基底核(線条体・淡蒼球・黒質)の線維結合と機能 \\
 &  & 痛覚、温度覚、触覚と深部感覚の受容機序と伝導路 \\
 &  & 視覚、聴覚・平衡覚、嗅覚、味覚の受容機序と伝導路 \\
 &  & 交感神経系と副交感神経系の中枢内局在、末梢分布、機能と伝達物質 \\
 &  & 内分泌および自律機能と関連づけた視床下部の構造と機能 \\
 &  & ストレス反応と本能・情動行動の発現機序 \\
 & 症候 & 頭痛 \\
 &  & めまい \\
 &  & けいれん \\
 &  & 意識障害 \\
 &  & 運動麻痺・筋力低下 \\
 &  & 歩行障害 \\
 &  & 感覚障害 \\
 &  & 認知障害 \\
 &  & 失語症・構音障害 \\
 &  & 振戦 \\
 &  & 小脳性・前庭性・感覚性運動失調障害 \\
 &  & 不随意運動(ミオクローヌス・舞踏運動・ジストニア・固定姿勢保持困難・アテトーシス・チック) \\
 &  & 頭蓋内圧亢進(急性・慢性) \\
 &  & 脳ヘルニア \\
 & 検査方法 & 脳・脊髄の画像検査(CT・MRI) \\
 &  & 神経系の電気生理学的検査(脳波検査、針筋電図検査、末梢神経伝導検査) \\
 & 特異的治療法 & 脳血管障害の急性期治療とリハビリテーション医療 \\
皮膚系 & 構造と機能 & 皮膚の組織構造 \\
 &  & 皮膚の細胞動態と角化の機構 \\
 &  & 皮膚の免疫防御能 \\
 & 症候 & 皮疹(紅斑・紫斑・色素斑・丘疹・結節・腫瘤・水疱・膿疱・嚢腫・びらん・潰瘍・毛細血管拡張・硬化・瘢痕・萎縮・鱗屑・痂皮・苔癬化・壊疽) \\
 &  & そう痒 \\
 &  & 粘膜疹 \\
 &  & 脱毛 \\
 & 検査方法 & 皮膚検査法(硝子圧法・皮膚描記法(Darier 徴候)・Nikolsky 現象・Tzanck 試験・光線テスト) \\
 &  & 皮膚アレルギー検査法(プリックテスト・皮内テスト・パッチテスト) \\
 &  & 微生物検査法(検体採取法・苛性カリ(KOH)直接検鏡法) \\
 &  & ダーモスコピー \\
 & 特異的治療法 & 外用療法 \\
 &  & 凍結療法 \\
 &  & 光線療法(PUVA療法) \\
運動器(筋骨格)系 & 構造と機能 & 骨・軟骨・関節・靱帯の構成と機能 \\
 &  & 頭頸部の構成 \\
 &  & 脊柱の構成と機能 \\
 &  & 四肢の骨格、主要筋群の運動と神経支配 \\
 &  & 骨盤の構成と性差 \\
 &  & 骨の成長と骨形成・吸収の機序 \\
 &  & 姿勢と体幹の運動にかかわる筋群 \\
 &  & 抗重力筋 \\
 & 症候 & 運動麻痺・筋力低下 \\
 &  & 関節痛・関節腫脹 \\
 &  & 腰背部痛 \\
 &  & 歩行障害 \\
 &  & 感覚障害 \\
 & 検査方法 & 筋骨格系の病態に即した徒手検査(四肢と脊柱の可動域検査・神経学的検査等) \\
 &  & 筋骨格系画像診断(単純エックス線撮影・CT・MRI・超音波検査・骨塩定量) \\
 &  & 関節液検査 \\
 & 特異的治療法 & 運動器疾患のリハビリテーション \\
 &  & 捻挫・骨折・脱臼の治療・処置 \\
循環器系 & 構造と機能 & 心臓の構造と分布する血管・神経、冠動脈の特長とその分布域 \\
 &  & 心筋細胞の微細構造と機能 \\
 &  & 心筋細胞の電気現象と心臓の興奮(刺激)伝導系 \\
 &  & 興奮収縮連関 \\
 &  & 体循環、肺循環と胎児・胎盤循環 \\
 &  & 大動脈と主な分枝(頭頸部、上肢、胸部、腹部、下肢)を図示し、分布域 \\
 &  & 主な静脈、門脈系と上・下大静脈系 \\
 &  & 毛細血管における物質・水分交換 \\
 &  & 胸管を経由するリンパの流れ \\
 &  & 心周期にともなう血行動態 \\
 &  & 心機能曲線と心拍出量の調節機序 \\
 &  & 主な臓器(脳、心臓、肺、腎臓)の循環調節 \\
 &  & 血圧調節の機序 \\
 &  & 体位や運動に伴う循環反応とその機序 \\
 & 症候 & 胸痛 \\
 &  & 腰背部痛 \\
 &  & 動悸 \\
 &  & 呼吸困難 \\
 &  & 咳・痰 \\
 &  & 浮腫 \\
 &  & 体重増加 \\
 &  & 意識障害 \\
 &  & 失神 \\
 &  & 胸水 \\
 & 検査方法 & 胸部単純エックス線撮影 \\
 &  & 心電図(安静時・運動負荷心電図・ホルタ―心電図) \\
 &  & 心臓超音波検査 \\
 &  & 心臓シンチグラフィ- \\
 &  & 冠動脈造影、冠動脈CT、MRI \\
 &  & 心カテーテル検査(心内圧・心機能・シャント率の測定) \\
 & 特異的治療法 & 虚血性心疾患に対する血行再建術(経皮的冠動脈形成術・ステント留置術・冠動脈バイパス術) \\
 &  & 不整脈に対する非薬物療法(カテーテルアブレーション・電気的除細動・ペースメーカー植え込み・植え込み型除細動器) \\
 &  & 心臓リハビリテーションなどの疾病管理プログラム \\
呼吸器系 & 構造と機能 & 気道の構造、肺葉・肺区域と肺門の構造 \\
 &  & 肺循環と体循環の違い \\
 &  & 縦隔と胸膜腔の構造 \\
 &  & 呼吸筋と呼吸運動の機序 \\
 &  & 肺気量分画、換気、死腔(換気力学(胸腔内圧、肺コンプライアンス、抵抗、クロージングボリューム(closing volume))) \\
 &  & 肺胞におけるガス交換と血流の関係 \\
 &  & 肺の換気と血流(換気血流比)が動脈血ガスにおよぼす影響(肺胞気-動脈血酸素分圧較差(alveolar-arterial oxygen difference (A-aDO2))) \\
 &  & 呼吸中枢を介する呼吸調節の機序 \\
 &  & 血液による酸素 と二酸化炭素 の運搬の仕組み \\
 &  & 気道と肺の防御機構(免疫学的・非免疫学的)と代謝機能 \\
 & 症候 & 胸痛 \\
 &  & 呼吸困難 \\
 &  & 咳・痰 \\
 &  & 血痰・喀血 \\
 &  & 喘鳴 \\
 &  & 胸部圧迫感 \\
 &  & 呼吸数・リズムの異常 \\
 &  & 胸水 \\
 & 検査方法 & 喀痰検査(喀痰細胞診・喀痰培養) \\
 &  & 胸水検査、胸膜生検 \\
 &  & 呼吸機能検査(スパイロメトリ・肺拡散能力・flow-volume曲線)、動脈血ガス分析、ポリソムノグラフィ、ピークフローメトリ \\
 &  & 画像検査(単純エックス線撮影・CT・MRI)、核医学検査(ポジトロン断層法(positron emission tomography (PET)) \\
 &  & 気管支内視鏡検査 \\
 & 特異的治療法 & 呼吸器理学療法・リハビリテーション \\
 &  & 酸素療法 \\
 &  & 人工換気 \\
消化器系 & 構造と機能 & 各消化器官の位置、形態と関係する血管 \\
 &  & 腹膜と臓器の関係 \\
 &  & 食道・胃・小腸・大腸の基本構造と部位による違い \\
 &  & 消化管運動の仕組み \\
 &  & 消化器官に対する自律神経の作用 \\
 &  & 肝の構造と機能 \\
 &  & 胃液の作用と分泌機序 \\
 &  & 胆汁の作用と胆嚢収縮の調節機序 \\
 &  & 膵外分泌系の構造と膵液の作用 \\
 &  & 小腸における消化・吸収の仕組み \\
 &  & 大腸における糞便形成と排便の仕組み \\
 &  & 主な消化管ホルモンの作用 \\
 &  & 歯、舌、唾液腺の構造と機能 \\
 &  & 咀しゃくと嚥下の機構 \\
 &  & 消化管の正常細菌叢(腸内細菌叢)の役割 \\
 & 症候 & 腹痛 \\
 &  & 悪心・嘔吐 \\
 &  & 食思(欲)不振 \\
 &  & 便秘 \\
 &  & 下痢 \\
 &  & 吐血 \\
 &  & 下血 \\
 &  & 腹部膨隆・腫瘤 \\
 &  & 黄疸 \\
 &  & 胸やけ \\
 &  & 肝腫大 \\
 & 検査方法 & 肝炎ウイルス検査 \\
 &  & 腫瘍マーカー(AFP・ CEA・ CA 19-9・ PIVKA-Ⅱなど) \\
 &  & 画像検査(単純エックス線撮影・CT・MRI) \\
 &  & 内視鏡検査 \\
 &  & 生検、細胞診 \\
 & 特異的治療法 & 経管・経腸栄養 \\
 &  & 内視鏡治療(止血・凝固・クリッピング・硬化療法など) \\
 &  & 血管内治療(動脈塞栓術など) \\
腎・尿路系(体液・電解質バランスを含む) & 構造と機能 & 体液の量と組成・浸透圧(小児と成人の違いを含めて) \\
 &  & 腎・尿路系の位置・形態と血管分布・神経支配 \\
 &  & 腎の機能の全体像やネフロン各部の構造と機能 \\
 &  & 腎糸球体における濾過の機序 \\
 &  & 尿細管各部における再吸収・分泌機構と尿の濃縮機序 \\
 &  & 水電解質、酸・塩基平衡の調節機構 \\
 &  & 腎で産生される又は腎に作用するホルモン・血管作動性物質(エリスロポエチン・ビタミンD、レニン・アンギオテンシンII、アルドステロン)の作用 \\
 &  & 蓄排尿の機序 \\
 & 症候 & 血尿 \\
 &  & タンパク尿 \\
 &  & 浮腫 \\
 &  & 脱水 \\
 &  & 尿量・排尿の異常 \\
 &  & 臨床症候の分類(急性腎炎症候群・慢性腎炎症候群・ネフローゼ症候群・急速進行性腎炎症候群・反復性または持続性血尿症候群) \\
 & 検査方法 & 糸球体濾過量(実測・推算)を含む腎機能検査法 \\
 &  & 腎・尿路系の画像診断(単純エックス線撮影・尿路造影・CT・MRI) \\
 &  & 腎生検の適応と禁忌 \\
 &  & 尿流動態検査 \\
 & 特異的治療法 & 腎代替療法(血液透析・腹膜透析・腎移植) \\
生殖機能 & 構造と機能 & 生殖腺の発生と性分化の過程 \\
 &  & 男性生殖器の発育の過程 \\
 &  & 男性生殖器の形態と機能 \\
 &  & 精巣の組織構造と精子形成の過程 \\
 &  & 陰茎の組織構造と勃起・射精の機序 \\
 &  & 女性生殖器の発育の過程 \\
 &  & 女性生殖器の形態と機能 \\
 &  & 性周期発現と排卵の機序 \\
 &  & 閉経の過程と疾病リスクの変化 \\
 & 症候 & 腹痛 \\
 &  & 腹部膨隆・腫瘤 \\
 &  &  \\
 &  &  \\
 &  & 月経異常 \\
 &  &  \\
 &  & 勃起不全 \\
 &  & 射精障害 \\
 &  & 精巣機能障害 \\
 &  & 不正性器出血 \\
 &  & 乳汁漏出症 \\
 &  & 腟分泌物(帯下)の増量 \\
 &  & 腟乾燥感 \\
 &  & 性交痛 \\
 &  &  \\
 & 検査方法 & 精巣と前立腺の画像検査法(尿路造影・CT・MRI)、超音波検査 \\
 &  & 血中ホルモン(卵胞刺激ホルモン(Follicle-Stimulating Hormone (FSH))、黄体形成ホルモン(luteinizing hormone (LH))、プロラクチン、ヒト絨毛性ゴナドトロピン(human chorionic gonadotropin (hCG))、エストロゲン、プロゲステロン)の測定 \\
 &  & 骨盤内臓器と腫瘍の画像診断(超音波断層法CT、MRI、子宮卵管造影(hysterosalpingography (HSG)) \\
 &  & 基礎体温測定 \\
 &  & 腟分泌物所見 \\
 & 特異的治療法 & 体外受精―胚移植(IVF-ET) \\
妊娠と分娩 & 構造と機能 & 妊娠・分娩・産褥での母体の解剖学的と生理学的変化 \\
 &  & 胎児・胎盤系の発達過程での機能・形態的変化 \\
 &  & 正常妊娠の経過(妊娠に伴う身体的変化を含む) \\
 &  & 正常分娩の経過 \\
 &  & 産褥の過程 \\
 &  & 育児に伴う母体の変化、精神問題および母子保健 \\
 & 症候 & 腹痛 \\
 &  & 悪心・嘔吐 \\
 &  & 腹部膨隆・腫瘤 \\
 &  & 性器出血 \\
 &  & 月経異常 \\
 & 検査方法 & 妊娠の検査(妊娠反応、超音波検査) \\
 &  & 妊娠中の検査(血液検査・出生前遺伝学的検査・羊水検査・分泌物検査・ノンストレステスト・超音波検査・超音波ドプラ法・羊水量) \\
 &  & 分娩の検査(超音波検査・胎児心拍数陣痛図) \\
 & 特異的治療法 & 妊娠時の薬物療法の注意点 \\
 &  & 人工妊娠中絶、鉗子・吸引分娩、帝王切開術の適応 \\
小児 & 症候 & 発熱 \\
 &  & 意識障害 \\
 &  & けいれん \\
 &  & 浮腫 \\
 &  & 発疹 \\
 &  & 咳・痰 \\
 &  & 呼吸困難 \\
 &  & 嚥下困難 \\
 &  & 腹痛 \\
 &  & 悪心・嘔吐 \\
 &  & 下血 \\
 &  & 便秘 \\
 &  & 下痢 \\
 &  & 黄疸 \\
 &  & 腹部膨隆・腫瘤 \\
 &  & リンパ節腫脹 \\
 &  & 尿量・排尿の異常 \\
 &  & 哺乳力低下 \\
 &  & 体重増加不良 \\
 &  & 活動性低下 \\
 & 検査方法 & 新生児マススクリーニング \\
 &  & 新生児聴覚スクリーニング \\
 &  & 乳幼児健康診査 \\
 & 特異的治療法 & 小児輸液療法 \\
 &  & 予防接種 \\
乳房 & 構造と機能 & 乳房の構造と機能 \\
 &  & 成長発達に伴う乳房の変化 \\
 &  & 乳汁分泌に関するホルモンの作用を説明できる。 \\
 & 症候 & 乳房腫瘤 \\
 &  & 異常乳汁分泌(血性乳頭分泌) \\
 &  & 乳房の腫脹・疼痛・変形 \\
 &  & 女性化乳房 \\
 & 検査方法 & 乳房腫瘤に対する画像診断(超音波検査・マンモグラフィ・MRI) \\
 &  & 乳房腫瘤に対する細胞・組織診断法 \\
 & 特異的治療法 & ※現時点で該当項目なし \\
内分泌・栄養・代謝系 & 構造と機能 & ホルモンの構造的分類、作用機序および分泌調節機能 \\
 &  & 視床下部ホルモン・下垂体ホルモンの名称、作用と相互関係 \\
 &  & 甲状腺と副甲状腺(上皮小体)から分泌されるホルモンの作用と分泌調節機構 \\
 &  & 副腎の構造と分泌されるホルモンの作用と分泌調節機構 \\
 &  & 膵島から分泌されるホルモンの作用 \\
 &  & 男性ホルモン・女性ホルモンの合成・代謝経路と作用 \\
 &  & 三大栄養素、ビタミン、微量元素の消化吸収と栄養素の生物学的利用効率 \\
 &  & 糖質・タンパク質・脂質の代謝経路と相互作用 \\
 &  & 血中ホルモン濃度に影響を与える因子およびホルモンの日内変動 \\
 & 症候 & 体重減少 \\
 &  & 体重増加 \\
 &  & 月経異常 \\
 &  & 低身長 \\
 &  & 甲状腺腫 \\
 &  & ホルモンの過剰または欠乏がもたらす身体症状 \\
 &  & エネルギー摂取の過剰または欠乏がもたらす身体症状 \\
 & 検査方法 & 血中・尿中ホルモン測定 \\
 &  & 内分泌機能検査、負荷試験 \\
 & 特異的治療法 & ※現時点でなし \\
眼・視覚系 & 構造と機能 & 眼球と付属器の構造 \\
 &  & 視覚情報の受容の仕組みと伝導路 \\
 &  & 眼球運動の仕組み \\
 &  & 対光反射、輻輳反射、角膜反射の機能 \\
 & 症候 & めまい \\
 &  & 頭痛 \\
 &  & 悪心・嘔吐 \\
 &  & 視力障害 \\
 &  & 視野異常 \\
 &  & 眼球運動障害 \\
 &  & 眼脂・眼の充血 \\
 &  & 飛蚊症 \\
 &  & 眼痛 \\
 & 検査方法 & 視力検査 \\
 &  & 視野検査 \\
 &  & 細隙灯顕微鏡検査 \\
 &  & 眼圧検査 \\
 &  & 眼底検査 \\
 & 特異的治療法 & レーザー治療 \\
耳鼻・咽喉・口腔系 & 構造と機能 & 外耳・中耳・内耳の構造 \\
 &  & 聴覚・平衡覚の受容のしくみと伝導路 \\
 &  & 口腔・鼻腔・咽頭・喉頭の構造 \\
 &  & 喉頭の機能と神経支配 \\
 &  & 眼球運動、姿勢制御と関連させた平衡感覚機構 \\
 &  & 味覚と嗅覚の受容のしくみと伝導路 \\
 & 症候 & めまい \\
 &  & 嚥下困難 \\
 &  & 気道狭窄 \\
 &  & 難聴 \\
 &  & 鼻出血 \\
 &  & 咽頭痛 \\
 &  & 開口障害 \\
 &  & 反回神経麻痺(嗄声) \\
 &  & 耳鳴 \\
 &  & 鼻閉 \\
 &  & 鼻漏 \\
 &  & 嗅覚障害 \\
 &  & いびき \\
 &  & 味覚障害 \\
 &  & 唾液分泌異常 \\
 &  & 口腔内異常 \\
 & 検査方法 & 聴力検査と平衡機能検査 \\
 &  & 味覚検査と嗅覚検査 \\
 &  & 耳鏡、鼻鏡、喉頭鏡、鼻咽腔・喉頭内視鏡 \\
 & 特異的治療法 & 補聴器・人工聴覚器 \\
 &  & 気管切開 \\
精神系 & 構造と機能 & ※神経系の項目を参照 \\
 & 症候 & 不安・抑うつ \\
 &  & 認知障害 \\
 &  & 意識障害 \\
 &  & 不眠 \\
 &  & 幻覚・妄想 \\
 &  & 心気症 \\
 & 検査方法 & 質問紙法 \\
 &  & Rorschachテスト \\
 &  & 簡易精神症状評価尺度(Brief Psychiatric Rating Scale (BPRS)) \\
 &  & Hamiltonうつ病評価尺度 \\
 &  & Beckのうつ病自己評価尺度 \\
 &  & 状態特性不安検査(State-Trait Anxiety Inventory  (STAI)) \\
 &  & Mini-Mental State Examination (MMSE) \\
 &  & 改訂長谷川式簡易知能評価スケール \\
 &  & 精神科診断分類法 \\
 & 特異的治療法 & 精神科面接 \\
 &  & 精神保健及び精神障害者福祉に関する法律、心神喪失者等医療観察法の適用場面 \\
 &  & コンサルテーション・リエゾン精神医学 \\
救急系(中毒・環境因子による疾患を含む) & 症候 & 地域の救急医療体制について病院前救護体制、メディカルコントロール、初期・二次・三次救急医療の概念を用いて説明できる \\
 &  & 意識障害 \\
 &  & 失神 \\
 &  & けいれん \\
 &  & 呼吸困難 \\
 &  & 胸痛 \\
 &  & 運動麻痺・筋力低下 \\
 &  & 腹痛 \\
 &  & 悪心・嘔吐 \\
 &  & 吐血 \\
 &  & ショック \\
 &  & 発熱 \\
 &  & 全身倦怠感 \\
 &  & 皮疹 \\
 &  & リンパ節腫脹 \\
 &  & 浮腫 \\
 &  & 呼吸困難 \\
 &  & 咳・痰 \\
 &  & 血尿 \\
 &  & 関節痛・関節腫脹 \\
 &  & 自己抗体の種類と臨床的意義を説明できる。 \\
 &  & 免疫抑制薬による治療 \\
 &  & リウマチ性疾患へのリハビリテーション \\
 &  & ショック \\
 &  & 発熱・高体温 \\
 &  & けいれん \\
 &  & 意識障害・失神 \\
 &  &  \\
 &  & 脱水 \\
 &  & 全身倦怠感 \\
 &  & 黄疸 \\
 &  & 発疹 \\
 &  & リンパ節腫脹 \\
 &  & 浮腫 \\
 &  & 胸水 \\
 &  & 胸痛・胸部圧迫感 \\
 &  & 呼吸困難・息切れ \\
 &  & 咳・痰 \\
 &  & 血痰・喀血 \\
 &  & 頭痛・頭重感 \\
 &  & 腹痛 \\
 &  & 悪心・嘔吐 \\
 &  & 便秘・下痢・血便 \\
 &  & 吐血・下血 \\
 &  & 血尿・タンパク尿 \\
 &  &  \\
 &  & 関節痛・関節腫脹 \\
 &  & 腰背部痛 \\
免疫・アレルギー & 症候 & 咽頭痛 \\
 &  & 発熱 \\
 &  & 食欲低下 \\
 &  & 体重減少 \\
 &  & 貧血 \\
 &  & リンパ節腫脹 \\
 &  & 心停止:心血管原性:急性心筋梗塞 \\
 &  & 心停止:心血管原性:急性大動脈解離 \\
 &  & 心停止:心血管原性:大動脈瘤破裂 \\
 & 検査方法 & 心停止:心血管原性:肺塞栓 \\
 & 特異的治療法 & 心停止:呼吸原性:気道閉塞 \\
 &  & 心停止:呼吸原性:緊張性気胸 \\
感染症 & 症候 & 心停止:呼吸原性:肺実質病変による低酸素血症 \\
 &  & 心停止:神経原性:重症頭部・脊髄外傷 \\
 &  & 心停止:神経原性:急性くも膜下出血 \\
 &  & 心停止:中毒・環境要因:中毒 \\
 &  & 心停止:中毒・環境要因:熱中症 \\
 &  & 心停止:中毒・環境要因:低体温症 \\
 &  & 心停止:電解質・酸塩基平衡異常:低・高カリウム血症 \\
 &  & 心停止:電解質・酸塩基平衡異常:アシドーシス \\
 &  & 心停止:電解質・酸塩基平衡異常:低血糖 \\
 &  & 中毒:食中毒 \\
 &  & 中毒:ガス中毒:一酸化炭素中毒 \\
 &  & 中毒:ガス中毒:硫化水素 \\
 &  & 中毒:ガス中毒:青酸ガス \\
 &  & 中毒:農薬:有機リン \\
 &  & 中毒:農薬:有機塩素 \\
 &  & 中毒:アルコール \\
 &  & 中毒:薬物:睡眠薬 \\
 &  & 中毒:薬物:向精神薬 \\
 &  & 中毒:薬物:解熱鎮痛薬 \\
 &  & 中毒:薬物:麻薬 \\
 &  & 中毒:薬物:覚醒剤 \\
 &  & 中毒:水銀 \\
 &  & 中毒:鉛 \\
 &  & 中毒:青酸 \\
 &  & 中毒:ヒ素 \\
 &  & 中毒:パラコート \\
腫瘍 & 症候 & 中毒:自然毒 \\
 &  & 中毒:腐食剤:酸 \\
 &  & 中毒:腐食剤:アルカリ \\
 &  & 中毒:腐食剤:フッ化水素 \\
 &  & 中毒:ボタン電池誤飲  \\
主要症候 & 症候 & 発熱 \\
 &  & 全身倦怠感 \\
 &  & 食思(欲)不振 \\
 &  & 体重減少 \\
 &  & 体重増加 \\
 &  & 意識障害 \\
 &  & 失神 \\
 &  & けいれん \\
 &  & めまい \\
 &  & 浮腫 \\
 &  & 発疹 \\
 &  & 咳・痰 \\
 &  & 血痰・喀血 \\
 &  & 呼吸困難 \\
 &  & 胸痛 \\
 &  & 動悸 \\
 &  & 嚥下困難 \\
 &  & 腹痛 \\
 &  & 悪心・嘔吐 \\
 &  & 吐血 \\
 &  & 下血 \\
 &  & 便秘 \\
 &  & 下痢 \\
 &  & 黄疸 \\
 &  & 腹部膨隆・腫瘤 \\
 &  & リンパ節腫脹 \\
 &  & 尿量・排尿の異常 \\
 &  & 血尿 \\
 &  & 月経異常 \\
 &  & 不安・抑うつ \\
 &  & 認知障害 \\
 &  & 頭痛 \\
 &  & 運動麻痺・筋力低下 \\
 &  & 歩行障害 \\
 &  & 感覚障害 \\
 &  & 腰背部痛 \\
 &  & 関節痛・関節腫脹 \\
\bottomrule
\end{xltabular}


